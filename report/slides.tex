\documentclass{beamer}
\usetheme{Frankfurt}

\usepackage{aliascnt}
\usepackage[spanish]{babel}
\usepackage{amsmath,amsthm,amssymb}
\usepackage{hyperref}
\usepackage{xcolor}
\usepackage{graphicx}
\usepackage{tikz}
\usepackage{booktabs}
\graphicspath{{assets/}}
\usetikzlibrary{shapes.geometric, arrows, positioning}

% Math environment
\setbeamertemplate{theorem}[ams style]
\deftranslation[to=spanish]{Theorem}{Teorema}
\deftranslation[to=spanish]{Corollary}{Corolario}
\deftranslation[to=spanish]{Lemma}{Lema}
\deftranslation[to=spanish]{Proof}{Demostración}
\deftranslation[to=spanish]{Definition}{Definición}
\deftranslation[to=spanish]{Example}{Ejemplo}
\theoremstyle{remark}
\newtheorem{remark}{Nota}

% Pseudocode environment
\usepackage[ruled,vlined]{algorithm2e}
\renewcommand{\algorithmcfname}{Algoritmo}
\renewcommand{\algorithmautorefname}{Algoritmo}
\renewcommand{\listalgorithmcfname}{Índice de algoritmos}

\title{Reconocimiento de rostros mediante descomposición en valores singulares (SVD)}
\author{Gonzalo Bordón\\ Emanuel Nicolás Herrador}
\institute{Facultad de Matemática, Astronomía, Física y Computación\\Universidad Nacional de Córdoba}
\titlegraphic{\includegraphics[width=5cm]{logo.png}}
\date{14 de Noviembre de 2025}

\newcommand{\appendixtitle}{
  \title{Apéndice}
  \subtitle{Material complementario}
}

\begin{document}
\frame{\titlepage}

\AtBeginSection[]{
  \begin{frame}
    \frametitle{Índice}
    \tableofcontents[currentsection]
  \end{frame}
}

\section{Introducción}
\subsection{Reconocimiento facial}
\begin{frame}{Reconocimiento facial}
  \begin{definition}
    Identificación o verificación de la identidad de una persona a partir de una
    imagen digital de su rostro.
  \end{definition}

  \begin{itemize}
    \item<2-> Una de las áreas más activas en el campo de visión por computadora e
      inteligencia artificial
    \item<2-> Aplicaciones en seguridad, control de acceso, interacción
      humano-computadora, sistemas biométricos, entre otros.
    \item<2-> \underline{Desafíos}: variaciones en iluminación, pose, expresión facial,
      edad y accesorios (gafas, gorros).
  \end{itemize}
\end{frame}

\subsection{Objetivos}
\begin{frame}{Objetivos}
  \begin{enumerate}
    \item Implementación del módulo de preprocesamiento de imágenes para estandarizarlas
      y comprimirlas.
    \item Implementación del algoritmo completo de reconocimiento facial basado en 
      SVD.
    \item Evaluación del rendimiento del sistema en datasets reales, incluyendo 
      análisis de precisión y exactitud, como matrices de confusión.
    \item Análisis de las propiedades de los espacios faciales generados, incluyendo 
      el estudio de los valores singulares y su relación con la capacidad de compresión 
      y reconstrucción de las imágenes 
  \end{enumerate}
\end{frame}

\section{Algoritmo}
\subsection{Preprocesamiento}
\begin{frame}{Preprocesamiento de imágenes}
  \alt<2->{
    \begin{itemize}
      \item \textbf{Compresión} usando SVD con $k=40$ valores singulares.
    \end{itemize}

    \begin{block}{Funcionamiento}
      Dada una imagen $I \in \mathbb{R}^{m\times n}$, sea $I = U\Sigma V^T$ su 
      descomposición SVD, entonces la imagen comprimida se reconstruye usando 
      solo los primeros $k$ valores singulares:
      \begin{equation*}
        \tilde{I} = \sum_{i=1}^k \sigma_i u_i v_i^T 
      \end{equation*}
    \end{block}
    \only<3->{
      \begin{block}{Ventajas}
        \begin{itemize}
          \item Espacio de almacenamiento: $m\times n$ a $k \cdot (m+n+1)$
          \item Estructura y formas principales se mantienen dado que los últimos valores 
            singulares capturan principalmente ruido
        \end{itemize}
      \end{block}
    }
  }{
    Pasos del preprocesamiento a realizar:
    \begin{itemize}
      \item \textbf{Conversión} a escala de grises.
      \item \textbf{Redimensionamiento} a $100\times 100$ pixeles.
      \item \textbf{Compresión} usando SVD con $k=40$ valores singulares.
      \item \textbf{Normalización} al rango $[0,1]$.
      \item \textbf{Vectorización} a $10000$ dimensiones.
    \end{itemize}
  }
\end{frame}
\begin{frame}{Comparación: Original vs Preprocesada}
  \alt<2->{
    \alt<3->{
      \begin{figure}
        \centering
        \includegraphics[width=0.4\textwidth]{celebrities_psnr.png}
        \hspace{1cm}
        \includegraphics[width=0.4\textwidth]{yale_psnr.png}
        \caption{\underline{Izquierda}: PSNR en dataset Celebrities. 
                 \underline{Derecha}: PSNR en dataset Yale.}
      \end{figure}
    }{
      \begin{block}{PSNR (Peak Signal-to-Noise Ratio)}
        Métrica que cuantifica la calidad de la imagen comprimida en comparación con la 
        original. Se define como:
        \begin{equation*}
          \text{PSNR} = 10 \cdot \log_{10}\left(\frac{MAX_I^2}{MSE}\right)
        \end{equation*}
        donde $MAX_I$ es el valor máximo posible de un píxel (255 para imágenes de 8 bits) 
        y $MSE$ es el error cuadrático medio entre la imagen original y la comprimida.
      \end{block}
      \begin{itemize}
        \item $> 30$ dB: Calidad aceptable, diferencias apenas perceptibles.
        \item $> 40$ dB: Alta calidad, diferencias casi indistinguibles.
        \item $< 20$ dB: Baja calidad, diferencias notables.
      \end{itemize}
    }
  }{
    \begin{figure}
      \centering
      \includegraphics[width=0.3\textwidth]{Angelina Jolie.jpg}
      \hspace{1cm}
      \includegraphics[width=0.36\textwidth]{Angelina Jolie - compressed.png}
      \caption{\underline{Izquierda}: Imagen original. \underline{Derecha}: Imagen preprocesada.}
    \end{figure}
  }
\end{frame}

\subsection{Reconocimiento facial}
\begin{frame}{Reconocimiento facial mediante SVD}
  \alt<5->{
    \begin{enumerate}
      \setcounter{enumi}{4}
      \item<5-> \textbf{Coordenadas faciales}: proyección de imágenes de entrenamiento 
        en el espacio de eigenfaces.
        \begin{equation*}
          \forall i \in [1, N],\quad x_i = U^T (f_i - \bar{f})
        \end{equation*}
      \item<6-> \textbf{Selección de umbrales}:
        \begin{itemize}
          \item \textbf{Umbral de reconocimiento facial} $\varepsilon_1$: distancia máxima para 
            considerar que una imagen es un rostro.
          \item \textbf{Umbral de identificación facial} $\varepsilon_0$: distancia máxima para
            considerar que una imagen pertenece a una persona conocida.
        \end{itemize}
      \item<7-> \textbf{Clasificación} de una imagen de prueba $f$: según distancia euclidiana en 
        el espacio de eigenfaces con las coordenadas faciales ($x_i$) y los umbrales.
      \end{enumerate}
    }{
      Los pasos del algoritmo de reconocimiento facial son:
      \begin{enumerate}
        \item<1-> \textbf{Conjunto de entrenamiento}: $S = [f_1\ f_2\ \dots\ f_N]$ donde 
          cada $f_i$ es una imagen vectorizada preprocesada.
        \item<2-> \textbf{Cálculo de la media facial}:
          \begin{equation*}
            \bar{f} = \frac{1}{N} \sum_{i=1}^N f_i
          \end{equation*}
        \item<3-> \textbf{Matriz de diferencias}: $A = [a_1\ a_2\ \dots\ a_N]$ donde 
          $a_i = f_i - \bar{f}$
        \item<4-> \textbf{Descomposición SVD} de $A$:
          \begin{equation*}
            A = U\Sigma V^T
          \end{equation*}
          donde las columnas de $U$ son los \textbf{eigenfaces}.
      \end{enumerate}
    }
\end{frame}
\begin{frame}{Diagrama de decisión}
  \begin{figure}
    \centering
    \scalebox{0.75}{
      \begin{tikzpicture}[
          node distance=1cm,
          decision/.style={diamond, draw, fill=blue!20, text width=2cm, text centered, minimum height=0.8cm, font=\scriptsize},
          process/.style={rectangle, draw, fill=green!20, text width=2.2cm, text centered, minimum height=0.8cm, font=\scriptsize},
          startstop/.style={rectangle, rounded corners, draw, fill=red!20, text width=2.2cm, text centered, minimum height=0.8cm, font=\scriptsize},
          arrow/.style={thick,->,>=stealth}
      ]

      \node [startstop] (start) {Nueva imagen $f$};
      \node [process, right of=start, xshift=2cm] (preprocess) {Preprocesar};
      \node [process, right of=preprocess, xshift=2cm] (coords) {Calcular $x$};
      \node [process, right of=coords, xshift=2cm] (distface) {$\epsilon_f = \|a - Ux\|_2$};
      \node [decision, below of=distface, yshift=-1.5cm] (isface) {$\epsilon_f \leq \epsilon_1$?};
      \node [process, below of=isface, yshift=-1.5cm] (notface) {No es cara};
      \node [process, left of=isface, xshift=-2cm] (distperson) {Cálculo de distancias\\ $\varepsilon_i=\|x-x_i\|_2$};
      \node [decision, left of=distperson, xshift=-2cm] (known) {$\min \varepsilon_i \leq \epsilon_0$?};
      \node [process, below of=known, yshift=-1.5cm] (unknown) {Desconocida};
      \node [startstop, left of=known, xshift=-2cm] (recognized) {Reconocida};

      \draw [arrow] (start) -- (preprocess);
      \draw [arrow] (preprocess) -- (coords);
      \draw [arrow] (coords) -- (distface);
      \draw [arrow] (distface) -- (isface);
      \draw [arrow] (isface) -- node[pos=0.5, right, font=\tiny] {No} (notface);
      \draw [arrow] (isface) -- node[pos=0.5, above, font=\tiny] {Sí} (distperson);

      \draw [arrow] (distperson) -- (known);
      \draw [arrow] (known) -- node[pos=0.5, right, font=\tiny] {No} (unknown);
      \draw [arrow] (known) -- node[pos=0.5, above, font=\tiny] {Sí} (recognized);

      \end{tikzpicture}
    }
    \caption{Diagrama de flujo del algoritmo de reconocimiento facial}
  \end{figure}
\end{frame}

\section{Resultados}
\subsection{Datasets}
\begin{frame}{Datasets utilizados}
  \begin{columns}
    \column{0.45\textwidth}
      \begin{block}{Dataset de Celebridades}
        \begin{itemize}
            \item 17 individuos
            \item $\sim$100 imágenes/persona
            \item Condiciones variables
        \end{itemize}
        \begin{figure}
          \centering
          \includegraphics[height=0.6\textwidth]{Angelina Jolie.jpg}
          \caption{Imagen representativa}
        \end{figure}
      \end{block}

    \column{0.45\textwidth}
      \begin{block}{Yale Face Database}
        \begin{itemize}
            \item 28 personas
            \item 584 imágenes/persona
            \item Condiciones controladas
        \end{itemize}
        \begin{figure}
          \centering
          \includegraphics[height=0.6\textwidth]{yaleB11.jpg}
          \caption{Imagen representativa}
        \end{figure}
      \end{block}
  \end{columns}
\end{frame}

\subsection{Rendimiento}
\begin{frame}{Rendimiento del sistema}
  \alt<2->{
    \begin{figure}
      \centering
      \begin{tabular}{c}
        \includegraphics[width=0.45\textwidth]{celebrities_classification_metrics.png} \\
        \includegraphics[width=0.45\textwidth]{yale_classification_metrics.png}
      \end{tabular}
      \caption{\footnotesize \underline{Arriba}: Celebrities. 
              \underline{Abajo}: Yale.}
    \end{figure}
  }{
  \begin{table}
    \centering
    \small
    \begin{tabular}{lcc}
    \toprule
    \textbf{Métrica} & \textbf{Celebridades} & \textbf{Yale} \\
    \midrule
    Condiciones & Variables & Controladas \\
    Imágenes/persona & $\sim$100 & 584 \\
    Exactitud & $15\%$ & $\mathbf{82.42\%}$ \\
    Precisión por persona & $0\%$ - $29\%$ & $> 59\%$ \\
    F1-Score & - & $> 60\%$ \\
    Top-2 accuracy & $27.9\%$ & $88.6\%$ \\
    Top-3 accuracy & $35\%$ & $91.2\%$ \\
    \bottomrule
    \end{tabular}
  \end{table}
  }
\end{frame}

\subsection{Análisis de resultados}
\begin{frame}{Análisis de resultados}
  \begin{block}{Dataset de Celebridades ($15\%$)}
    \begin{itemize}
      \item \textbf{Pocas imágenes por persona} ($\sim$100 vs. 584)
      \item \textbf{Condiciones no controladas}: iluminación, pose, expresión variables
      \item \textbf{Falta de alineación facial} adecuada
      \item \textbf{Variabilidad alta} entre imágenes de la misma persona
    \end{itemize}
  \end{block}

  \begin{block}{Yale Face Database ($82.42\%$)}
    \begin{itemize}
      \item \textbf{Más imágenes por persona} (584) → espacio facial más robusto
      \item \textbf{Condiciones controladas}: iluminación y pose consistentes
      \item \textbf{Mejor alineación facial}
      \item \textbf{Preprocesamiento uniforme}
    \end{itemize}
  \end{block}
\end{frame}

\section{Conclusiones y posibles mejoras}
\subsection{Conclusiones}
\begin{frame}{Conclusiones Principales}
  \begin{block}{Resultado Clave}
    El método SVD es \textbf{viable para reconocimiento facial} cuando se cuenta con:
    \begin{itemize}
      \item Suficientes imágenes por persona
      \item Condiciones de captura controladas
      \item Buen preprocesamiento y alineación
    \end{itemize}
  \end{block}

  \begin{block}{Factor Determinante}
    La \textbf{calidad y cantidad de datos de entrenamiento} son factores críticos
    para el éxito del método.
  \end{block}
\end{frame}

\subsection{Mejoras posibles}
\begin{frame}{Mejoras Posibles}
  \begin{itemize}
    \item \textbf{Preprocesamiento mejorado}: Ecualización de histograma adaptativa,
      normalización de iluminación, alineación facial precisa basada en puntos clave.
    \item \textbf{Selección adaptativa de umbrales}: Métodos automáticos basados en distribución
      de distancias (percentiles, análisis estadístico).
    \item \textbf{Reducción de dimensionalidad selectiva}: Usar solo los primeros $k$
      valores singulares más importantes.
    \item \textbf{Aumento de datos}: Generar variaciones sintéticas (rotaciones,
      cambios de iluminación) para robustecer el espacio facial.
  \end{itemize}
\end{frame}

\end{document}
