\documentclass[12pt,spanish]{beamer}

% Tema y configuración
\usetheme{Madrid}
\usecolortheme{default}

% Paquetes necesarios
\usepackage[font=scriptsize]{caption}
\usepackage[utf8]{inputenc}
\usepackage[spanish]{babel}
\usepackage{amsmath}
\usepackage{amsfonts}
\usepackage{amssymb}
\usepackage{graphicx}
\usepackage{booktabs}
\usepackage{tikz}
\usetikzlibrary{shapes.geometric, arrows, positioning}

\graphicspath{{./assets/}}

% Información del documento
\title{Reconocimiento de rostros mediante\\Descomposición en Valores Singulares (SVD)}
\author{Gonzalo Bordón \\ Emanuel Nicolás Herrador}
\date{\today}
\institute{Universidad Nacional de Córdoba}

\begin{document}

% Portada
\begin{frame}
\titlepage
\end{frame}

% ============================================
% PREPROCESAMIENTO
% ============================================
\section{Preprocesamiento}

\begin{frame}{Preprocesamiento de Imágenes}
\begin{enumerate}
    \item \textbf{Conversión a escala de grises}
    \item \textbf{Redimensionamiento} a 100×100 píxeles
    \item \textbf{Compresión usando SVD} con $k=40$ valores singulares
    \item \textbf{Normalización} al rango [0, 1]
    \item \textbf{Vectorización} a 10,000 dimensiones
\end{enumerate}
\end{frame}

\begin{frame}{Compresión de Imágenes con SVD}
\begin{block}{¿Cómo funciona?}
Dada una imagen $I \in \mathbb{R}^{m \times n}$, su descomposición SVD es:
\begin{equation*}
I = U\Sigma V^T
\end{equation*}
La imagen comprimida se reconstruye usando solo los primeros $k$ valores singulares:
\begin{equation*}
I_{comp} = \sum_{i=1}^{k} \sigma_i u_i v_i^T
\end{equation*}
\end{block}

\begin{block}{Ventajas}
\begin{itemize}
    \item Reduce significativamente el espacio de almacenamiento
    \item Los últimos valores singulares capturan principalmente ruido
    \item La estructura y formas principales se mantienen
\end{itemize}
\end{block}
\end{frame}

\begin{frame}{Comparación: Original vs Comprimida}
\begin{figure}
\centering
\includegraphics[width=0.9\textwidth]{compression_comparison.png}
\caption{Comparación visual: imagen original (izq.) vs comprimida con $k=40$ (der.)}
\end{figure}
\end{frame}

% ============================================
% ALGORITMO
% ============================================
\section{Algoritmo}

\begin{frame}{Algoritmo SVD - Pasos 1-4}
\begin{block}{Paso 1: Conjunto de Entrenamiento}
$S = [f_1, f_2, \ldots, f_N]$ donde cada $f_i$ es una imagen vectorizada
\end{block}

\begin{block}{Paso 2: Imagen Media}
$\bar{f} = \frac{1}{N}\sum_{i=1}^N f_i$
\end{block}

\begin{block}{Paso 3: Matriz de Diferencias}
$A = [a_1, a_2, \ldots, a_N]$, donde $a_i = f_i - \bar{f}$
\end{block}

\begin{block}{Paso 4: Descomposición SVD}
$A = U\Sigma V^T$ \\ Las columnas de $U$ son los \textbf{eigenfaces}
\end{block}
\end{frame}

\begin{frame}{Algoritmo SVD - Pasos 5-7}
\begin{block}{Paso 5: Coordenadas de Entrenamiento}
Para cada imagen $f_i$: $x_i = U^T (f_i - \bar{f})$
\end{block}

\begin{block}{Paso 6: Selección de Umbrales}
\begin{itemize}
    \item $\epsilon_1$: umbral para determinar si es una cara
    \item $\epsilon_0$: umbral para determinar si pertenece a un individuo conocido
\end{itemize}
\end{block}

\begin{block}{Paso 7: Clasificación}
Calcular coordenadas y distancias para nueva imagen
\end{block}
\end{frame}

\begin{frame}{Diagrama de Decisión}
\begin{figure}
\centering
\scalebox{0.75}{
\begin{tikzpicture}[
    node distance=1cm,
    decision/.style={diamond, draw, fill=blue!20, text width=2cm, text centered, minimum height=0.8cm, font=\scriptsize},
    process/.style={rectangle, draw, fill=green!20, text width=2.2cm, text centered, minimum height=0.8cm, font=\scriptsize},
    startstop/.style={rectangle, rounded corners, draw, fill=red!20, text width=2.2cm, text centered, minimum height=0.8cm, font=\scriptsize},
    arrow/.style={thick,->,>=stealth}
]

% Nodos
\node [startstop] (start) {Nueva imagen $f$};
\node [process, right of=start, xshift=2cm] (preprocess) {Preprocesar};
\node [process, right of=preprocess, xshift=2cm] (coords) {Calcular $x$};
\node [process, right of=coords, xshift=2cm] (distface) {$\epsilon_f = \|a - Ux\|$};
\node [decision, below of=distface, yshift=-1.5cm] (isface) {$\epsilon_f \leq \epsilon_1$?};
\node [process, below of=isface, yshift=-1.5cm] (notface) {No es cara};
\node [process, left of=isface, xshift=-2cm] (distperson) {Calculo de distancias a personas};
\node [decision, left of=distperson, xshift=-2cm] (known) {$\min \leq \epsilon_0$?};
\node [process, below of=known, yshift=-1.5cm] (unknown) {Desconocida};
\node [startstop, left of=known, xshift=-2cm] (recognized) {Reconocida};

% Flechas
\draw [arrow] (start) -- (preprocess);
\draw [arrow] (preprocess) -- (coords);
\draw [arrow] (coords) -- (distface);
\draw [arrow] (distface) -- (isface);
\draw [arrow] (isface) -- node[anchor=south, font=\tiny] {No} (notface);
\draw [arrow] (isface) -- node[anchor=east, font=\tiny] {Sí} (distperson);
\draw [arrow] (distperson) -- (known);
\draw [arrow] (known) -- node[anchor=south, font=\tiny] {No} (unknown);
\draw [arrow] (known) -- node[anchor=east, font=\tiny] {Sí} (recognized);

\end{tikzpicture}
}
\caption{Diagrama de flujo del algoritmo de reconocimiento facial}
\end{figure}
\end{frame}

% ============================================
% RESULTADOS
% ============================================
\section{Resultados}

\begin{frame}{Datasets y Resultados}
\begin{columns}
\column{0.45\textwidth}
\begin{block}{Dataset de Celebridades}
\begin{itemize}
    \item 17 individuos
    \item $\sim$100 imágenes/persona
    \item Condiciones variables
\end{itemize}
\begin{figure}
\centering
\includegraphics[height=0.6\textwidth]{Angelina Jolie.jpg}
\caption{Imagen representativa del dataset de celebridades}

\end{figure}
\end{block}

\column{0.45\textwidth}
\begin{block}{Yale Face Database}
\begin{itemize}
    \item 28 personas
    \item 584 imágenes/persona
    \item Condiciones controladas
\end{itemize}
\begin{figure}
\centering
\includegraphics[height=0.6\textwidth]{yaleB11.jpg}
\caption{Imagen representativa del Yale Face Database}

\end{figure}
\end{block}
\end{columns}
\end{frame}

\begin{frame}{Comparación de Resultados}
\begin{table}
\centering
\small
\begin{tabular}{lcc}
\toprule
\textbf{Métrica} & \textbf{Celebridades} & \textbf{Yale} \\
\midrule
Condiciones & Variables & Controladas \\
Imágenes/persona & $\sim$100 & 584 \\
Exactitud & $0.15\%$ & $\mathbf{82.42\%}$ \\
Precisión por persona & $0\%$ - $29\%$ & $> 59\%$ \\
F1-Score & - & $> 60\%$ \\
Top-2 accuracy & $27.9\%$ & $88.6\%$ \\
Top-3 accuracy & $35\%$ & $91.2\%$ \\
\bottomrule
\end{tabular}
\end{table}
\end{frame}

% ============================================
% CONCLUSIONES
% ============================================
\section{Conclusiones}

\begin{frame}{¿Por qué la diferencia de rendimiento?}
\begin{block}{Dataset de Celebridades ($0.15\%$)}
\begin{itemize}
    \item \textbf{Pocas imágenes por persona} ($\sim$100 vs. 584)
    \item \textbf{Condiciones no controladas}: iluminación, pose, expresión variables
    \item \textbf{Falta de alineación facial} adecuada
    \item \textbf{Variabilidad alta} entre imágenes de la misma persona
\end{itemize}
\end{block}

\begin{block}{Yale Face Database ($82.42\%$)}
\begin{itemize}
    \item \textbf{Más imágenes por persona} (584) → espacio facial más robusto
    \item \textbf{Condiciones controladas}: iluminación y pose consistentes
    \item \textbf{Mejor alineación facial}
    \item \textbf{Preprocesamiento uniforme}
\end{itemize}
\end{block}
\end{frame}

\begin{frame}{Conclusiones Principales}
\begin{block}{Resultado Clave}
El método SVD es \textbf{viable para reconocimiento facial} cuando se cuenta con:
\begin{itemize}
    \item Suficientes imágenes por persona
    \item Condiciones de captura controladas
    \item Buen preprocesamiento y alineación
\end{itemize}
\end{block}

\begin{block}{Factor Determinante}
La \textbf{calidad y cantidad de datos de entrenamiento} son factores críticos para el éxito del método
\end{block}
\end{frame}

\begin{frame}{Mejoras Posibles}
\begin{itemize}
    \item \textbf{Preprocesamiento mejorado}: Ecualización de histograma adaptativa, normalización de iluminación, alineación facial precisa basada en puntos clave
    \item \textbf{Selección adaptativa de umbrales}: Métodos automáticos basados en distribución de distancias (percentiles, análisis estadístico)
    \item \textbf{Reducción de dimensionalidad selectiva}: Usar solo los primeros $k$ valores singulares más importantes
    \item \textbf{Aumento de datos}: Generar variaciones sintéticas (rotaciones, cambios de iluminación) para robustecer el espacio facial
\end{itemize}
\end{frame}

% ============================================
% REFERENCIAS
% ============================================
\section{Referencias}

\begin{frame}{Referencias}
\begin{thebibliography}{99}
\bibitem{paper}
Singular Value Decomposition Applied To Digital Image Processing. Lijie Cao.

\bibitem{celebrity_dataset}
Celebrity Face Image Dataset. Kaggle.
\url{https://www.kaggle.com/datasets/vishesh1412/celebrity-face-image-dataset}

\bibitem{yale_dataset}
Extended Yale B Cropped Full. Kaggle.
\url{https://www.kaggle.com/datasets/jensdhondt/extendedyaleb-cropped-full/data}
\end{thebibliography}
\end{frame}

\begin{frame}
\centering
\Huge ¡Gracias por su atención!
\end{frame}

\end{document}
