\documentclass[12pt,a4paper]{article}

% Paquetes necesarios
\usepackage[utf8]{inputenc}
\usepackage[spanish]{babel}
\usepackage{amsmath}
\usepackage{amsfonts}
\usepackage{amssymb}
\usepackage{graphicx}
\usepackage{float}
\usepackage{hyperref}
\usepackage{geometry}
\usepackage{algorithm}
\usepackage{algorithmic}
\usepackage{caption}
\usepackage{subcaption}
\usepackage{booktabs}
\usepackage{multirow}

% Configuración de página
\geometry{margin=2.5cm}

% Configuración de hipervínculos
\hypersetup{
    colorlinks=true,
    linkcolor=blue,
    filecolor=magenta,      
    urlcolor=cyan,
}

% Información del documento
\title{}
\author{}
\date{}

\begin{document}

% Portada
\maketitle

% Resumen
\begin{abstract}

\end{abstract}

% Tabla de contenidos
\tableofcontents
\newpage

% Lista de figuras
\listoffigures
\newpage

% Lista de tablas
\listoftables
\newpage

% Introducción
\section{Introducción}
\subsection{Contexto}

El reconocimiento facial es una de las áreas más activas en el campo de la visión por computadora y la inteligencia artificial. 
Esta tecnología ha experimentado un crecimiento significativo en las últimas décadas debido a sus múltiples aplicaciones en seguridad, control de acceso, interacción humano-computadora y sistemas biométricos, entre otros.

El problema fundamental del reconocimiento facial consiste en identificar o verificar la identidad de una persona a partir de una imagen digital de su rostro. 
Este desafío presenta múltiples complicaciones, incluyendo variaciones en iluminación, pose, expresión facial, edad y la presencia de accesorios como gafas o barba. 
A pesar de estas dificultades, el reconocimiento facial ha demostrado ser una herramienta poderosa y confiable para la identificación de individuos.

Diversos enfoques han sido propuestos para abordar este problema, desde métodos basados en características geométricas hasta técnicas más modernas que utilizan aprendizaje profundo. 
Sin embargo, los métodos basados en análisis estadístico y álgebra lineal, como la Descomposición en Valores Singulares (SVD) y el Análisis de Componentes Principales (PCA), han demostrado ser particularmente efectivos y computacionalmente eficientes.

La Descomposición en Valores Singulares es una técnica matemática fundamental que permite descomponer una matriz en componentes que capturan las características más importantes de los datos. 
Cuando se aplica al reconocimiento facial, SVD permite construir un 'espacio facial' donde cada rostro puede ser representado como una combinación lineal de 'caras base' o eigenfaces. 
Este enfoque no solo facilita el reconocimiento, sino que también permite la compresión de imágenes y la reducción de dimensionalidad, manteniendo la información esencial para la identificación.

\subsection{Objetivos}

El objetivo principal de este trabajo es implementar y evaluar un sistema de reconocimiento facial utilizando la Descomposición en Valores Singulares (SVD), siguiendo la metodología propuesta en las notas académicas de la Universidad Nacional de Córdoba.

Los objetivos específicos incluyen:

\begin{itemize}
    \item Implementar el algoritmo completo de reconocimiento facial basado en SVD, siguiendo los siete pasos fundamentales del proceso: obtención del conjunto de entrenamiento, cálculo de la imagen media, formación de la matriz de diferencias, cálculo de la descomposición SVD, cálculo de coordenadas de entrenamiento, selección de umbrales y clasificación de nuevas imágenes.
    
    \item Evaluar el rendimiento del sistema utilizando un dataset real de imágenes faciales de múltiples individuos, incluyendo análisis de precisión, matrices de confusión y métricas de distancia.
    
    \item Analizar las propiedades del espacio facial generado mediante SVD, incluyendo el estudio de los valores singulares y su relación con la capacidad de compresión y reconstrucción de imágenes.
    
    \item Examinar la efectividad del método en términos de precisión de reconocimiento y comparar los resultados obtenidos con las expectativas teóricas del algoritmo.
    
    \item Documentar el proceso de implementación y los resultados obtenidos.
\end{itemize}

\subsection{Alcance del trabajo}

Este trabajo se enfoca en la implementación y evaluación del algoritmo SVD para reconocimiento facial según la metodología establecida en las notas académicas de referencia. El alcance del proyecto incluye:

\begin{itemize}
    \item \textbf{Implementación del algoritmo}: Se implementa el algoritmo completo de reconocimiento facial basado en SVD, incluyendo todas las etapas desde el preprocesamiento de imágenes hasta la clasificación final.
    
    \item \textbf{Dataset de evaluación}: Se utiliza un dataset real de imágenes faciales de múltiples celebridades, permitiendo evaluar el rendimiento del sistema con datos reales y variados.
    
    \item \textbf{Análisis de resultados}: Se realiza un análisis cuantitativo del rendimiento del sistema mediante métricas estándar de reconocimiento facial, incluyendo precisión, matrices de confusión y análisis de distancias.
    
    \item \textbf{Estudio teórico}: Se presenta el marco teórico necesario para comprender el funcionamiento del algoritmo SVD y su aplicación al reconocimiento facial.
\end{itemize}

El trabajo no incluye comparaciones con otros métodos de reconocimiento facial, ni optimizaciones avanzadas del algoritmo. Tampoco se aborda el problema de detección de caras en imágenes complejas, asumiendo que las imágenes de entrada ya contienen rostros detectados y preprocesados. El enfoque se mantiene en la implementación y evaluación del método SVD clásico según la metodología propuesta.

% Marco teórico
\section{Marco Teórico}
\subsection{Descomposición en Valores Singulares (SVD)}
\subsubsection{Definición matemática}

La Descomposición en Valores Singulares (SVD) es una factorización matricial fundamental en álgebra lineal. Dada una matriz $A \in \mathbb{R}^{m \times n}$ con rango $r$, la SVD permite descomponer $A$ en tres matrices:

\begin{equation}
A = U\Sigma V^T
\end{equation}

donde:
\begin{itemize}
    \item $U \in \mathbb{R}^{m \times m}$ es una matriz ortogonal cuyas columnas $u_i$ forman un conjunto ortonormal de vectores singulares izquierdos.
    \item $\Sigma \in \mathbb{R}^{m \times n}$ es una matriz que contiene los valores singulares $\sigma_1, \sigma_2, \ldots, \sigma_k$ en su diagonal, ordenados de forma decreciente: $\sigma_1 \geq \sigma_2 \geq \cdots \geq \sigma_r > 0$ y $\sigma_{r+1} = \sigma_{r+2} = \cdots = \sigma_k = 0$ con $k = \min(m, n)$.
    \item $V \in \mathbb{R}^{n \times n}$ es una matriz ortogonal cuyas columnas $v_i$ forman un conjunto ortonormal de vectores singulares derechos.
\end{itemize}

\subsubsection{Propiedades de SVD}

Las principales propiedades de la SVD incluyen:

\begin{itemize}
    \item Los valores singulares $\sigma_i$ son únicos, mientras que las matrices $U$ y $V$ pueden no serlo.
    \item Los vectores $v_i$ son los vectores propios de $A^TA$, ya que $A^TA = V\Sigma^T\Sigma V^T$.
    \item Los vectores $u_i$ son los vectores propios de $AA^T$, ya que $AA^T = U\Sigma\Sigma^TU^T$.
    \item Si $A$ tiene rango $r$, entonces $v_1, v_2, \ldots, v_r$ forman una base ortonormal para el espacio imagen de $A^T$, y $u_1, u_2, \ldots, u_r$ forman una base ortonormal para el espacio imagen de $A$.
    \item El rango de la matriz $A$ es igual al número de valores singulares distintos de cero.
    \item La matriz $A$ puede expresarse como una suma de productos exteriores: $A = \sum_{i=1}^r \sigma_i u_i v_i^T$.
\end{itemize}

\subsection{Reconocimiento Facial}
\subsubsection{Enfoques tradicionales}

El reconocimiento facial ha sido abordado mediante diversos enfoques a lo largo de los años. Los métodos tradicionales incluyen:

\begin{itemize}
    \item \textbf{Métodos geométricos}: Basados en la medición de distancias y relaciones entre características faciales (ojos, nariz, boca).
    \item \textbf{Métodos basados en plantillas}: Comparación directa de imágenes mediante correlación o distancias.
    \item \textbf{Métodos estadísticos}: Utilizan técnicas de reducción de dimensionalidad como Análisis de Componentes Principales (PCA) o Análisis Discriminante Lineal (LDA).
    \item \textbf{Métodos basados en SVD}: Construyen un espacio facial mediante descomposición en valores singulares.
\end{itemize}

Estos métodos tradicionales, aunque menos sofisticados que las técnicas modernas de aprendizaje profundo, ofrecen ventajas en términos de interpretabilidad, eficiencia computacional y requerimientos de datos.

\subsubsection{Espacio facial}

El concepto de espacio facial se refiere a un subespacio de menor dimensionalidad donde se pueden representar las imágenes faciales. En este espacio, cada rostro se representa como un punto o vector, y rostros similares se encuentran cerca unos de otros.

El espacio facial se construye a partir de un conjunto de imágenes de entrenamiento, identificando las direcciones principales de variación entre las diferentes caras. Estas direcciones principales, conocidas como eigenfaces o caras base, forman una base ortonormal del espacio facial.

La ventaja principal del espacio facial es que permite reducir significativamente la dimensionalidad del problema: en lugar de trabajar con imágenes de miles de píxeles, se trabaja con coordenadas en un espacio de dimensión mucho menor, generalmente igual al número de imágenes de entrenamiento o menor.

% Metodología
\section{Metodología}
\subsection{Dataset}
\subsubsection{Descripción del dataset}

El dataset utilizado en este trabajo consiste en imágenes faciales de múltiples individuos, más especificamente, de diferentes celebridades, cada una representada por múltiples fotografías que capturan variaciones en pose, iluminación y expresión facial.

Para el entrenamiento del modelo, se utilizaron imágenes de 17 individuos diferentes, con 100 imágenes faciales por individuo. Las imágenes fueron organizadas en directorios separados por individuo, facilitando la organización y el procesamiento del dataset.

\subsubsection{Preprocesamiento}

El preprocesamiento de las imágenes es un paso crucial para garantizar el buen funcionamiento del algoritmo SVD. El proceso de preprocesamiento incluye las siguientes etapas:

\begin{itemize}
    \item \textbf{Conversión a escala de grises}: Las imágenes se cargan en escala de grises para simplificar el procesamiento y reducir la dimensionalidad de los datos.
    \item \textbf{Redimensionamiento}: Todas las caras extraídas se redimensionan a un tamaño estándar de 100x100 píxeles para mantener consistencia dimensional.
    \item \textbf{Normalización}: Los valores de píxeles se normalizan al rango [0, 1] dividiendo por 255.
    \item \textbf{Vectorización}: Cada imagen de 100x100 píxeles se convierte en un vector de 10,000 dimensiones para su procesamiento.
\end{itemize}

Este preprocesamiento asegura que todas las imágenes tengan el mismo formato y características similares, lo cual es esencial para el correcto funcionamiento del algoritmo SVD.

\subsection{Implementación del Algoritmo SVD}

La implementación del algoritmo de reconocimiento facial basado en SVD sigue una metodología específica que consta de siete pasos principales:

\subsubsection{Paso 1: Obtención del conjunto de entrenamiento}

Se recopila un conjunto $S$ con $N$ imágenes faciales de individuos conocidos. Cada imagen $f_i$ se representa como un vector columna de $M$ píxeles (donde $M = 100 \times 100 = 10,000$). El conjunto de entrenamiento se organiza como una matriz $S \in \mathbb{R}^{M \times N}$ donde cada columna representa una imagen facial:

\begin{equation}
S = [f_1, f_2, \ldots, f_N]
\end{equation}

\subsubsection{Paso 2: Cálculo de la imagen media}

Se calcula la imagen media $\bar{f}$ del conjunto de entrenamiento mediante:

\begin{equation}
\bar{f} = \frac{1}{N}\sum_{i=1}^N f_i
\end{equation}

Esta imagen media representa el 'rostro promedio' del conjunto de entrenamiento y se utiliza como referencia para calcular las diferencias individuales.

\subsubsection{Paso 3: Formación de la matriz A}

Se forma la matriz $A$ que contiene las diferencias de cada imagen con respecto a la imagen media:

\begin{equation}
a_i = f_i - \bar{f}
\end{equation}

\begin{equation}
A = [a_1, a_2, \ldots, a_N]
\end{equation}

Esta matriz $A \in \mathbb{R}^{M \times N}$ captura las variaciones de cada rostro respecto al promedio.

\subsubsection{Paso 4: Cálculo de SVD}

Se calcula la descomposición en valores singulares de la matriz $A$:

\begin{equation}
A = U\Sigma V^T
\end{equation}

donde:
\begin{itemize}
    \item $U \in \mathbb{R}^{M \times M}$ contiene los vectores singulares izquierdos (eigenfaces).
    \item $\Sigma \in \mathbb{R}^{M \times N}$ contiene los valores singulares en su diagonal.
    \item $V \in \mathbb{R}^{N \times N}$ contiene los vectores singulares derechos.
\end{itemize}

Los vectores $u_i$ (columnas de $U$) forman una base ortonormal del espacio facial y se conocen como eigenfaces.

\subsubsection{Paso 5: Cálculo de coordenadas de entrenamiento}

Para cada individuo conocido, se calculan sus coordenadas $x_i$ en el espacio facial. Dada una imagen de entrenamiento $f_i$, sus coordenadas se calculan mediante:

\begin{equation}
x_i = [u_1, u_2, \ldots, u_r]^T (f_i - \bar{f})
\end{equation}

donde $r$ es el rango de la matriz $A$ (número de valores singulares distintos de cero). Estas coordenadas representan la posición de cada cara en el espacio facial.

\subsubsection{Paso 6: Selección de umbrales}

Se eligen dos umbrales críticos para el proceso de reconocimiento:

\begin{itemize}
    \item $\epsilon_1$: Umbral para determinar si una imagen contiene una cara. Si la distancia de una imagen al espacio facial es mayor que $\epsilon_1$, se considera que la imagen no es una cara.
    \item $\epsilon_0$: Umbral para determinar si una cara pertenece a un individuo conocido. Si la distancia mínima a cualquier individuo conocido es mayor que $\epsilon_0$, se clasifica como "cara desconocida".
\end{itemize}

En este trabajo se utilizaron los valores $\epsilon_1 = 0.3$ y $\epsilon_0 = 0.2$, seleccionados como valores por defecto del sistema.

\subsubsection{Paso 7: Clasificación de nuevas imágenes}

Para clasificar una nueva imagen $f$:

\begin{enumerate}
    \item Se calcula su diferencia con la imagen media: $a = f - \bar{f}$.
    \item Se calculan sus coordenadas en el espacio facial: $x = [u_1, u_2, \ldots, u_r]^T a$.
    \item Se calcula la distancia al espacio facial: $\epsilon_f = \|a - Ux\|_2$.
    \item Si $\epsilon_f > \epsilon_1$, la imagen no es una cara.
    \item Si $\epsilon_f \leq \epsilon_1$, se calculan las distancias a cada individuo conocido: $\epsilon_i = \|x - x_i\|_2$.
    \item Si $\min_i \epsilon_i > \epsilon_0$, se clasifica como "cara desconocida".
    \item Si $\min_i \epsilon_i \leq \epsilon_0$, se clasifica como el individuo con menor distancia.
\end{enumerate}

\subsection{Evaluación}
\subsubsection{Métricas utilizadas}

Para evaluar el rendimiento del sistema de reconocimiento facial, se utilizaron las siguientes métricas:

\begin{itemize}
    \item \textbf{Precisión de reconocimiento}: Proporción de imágenes correctamente clasificadas sobre el total de imágenes evaluadas. Se calcula como:
    \begin{equation}
    \text{Precisión} = \frac{\text{Clasificaciones correctas}}{\text{Total de imágenes}}
    \end{equation}
    
    \item \textbf{Precisión por persona}: Precisión individual para cada persona en el dataset, permitiendo identificar qué individuos son más fáciles o difíciles de reconocer.
    
    \item \textbf{Matriz de confusión}: Tabla que muestra cómo se clasificaron las imágenes, indicando cuántas imágenes de cada persona fueron clasificadas correctamente y cuántas fueron confundidas con otras personas.
    
    \item \textbf{Análisis de distancias}: Estudio de las distancias calculadas entre imágenes, incluyendo distancias al espacio facial y distancias entre personas conocidas.
\end{itemize}

\subsubsection{Procedimiento de evaluación}

El procedimiento de evaluación se realizó de la siguiente manera:

\begin{enumerate}
    \item Se dividió el dataset en conjunto de entrenamiento y conjunto de prueba. Las imágenes utilizadas para entrenar el modelo no se utilizaron para evaluar su rendimiento.
    \item Se entrenó el modelo SVD utilizando el conjunto de entrenamiento siguiendo los pasos 1-6 descritos anteriormente.
    \item Se evaluó el modelo utilizando imágenes del conjunto de prueba, calculando para cada imagen:
    \begin{itemize}
        \item Si fue correctamente reconocida (coincide con su etiqueta real).
        \item La distancia mínima al individuo correcto.
        \item Las distancias a todos los individuos conocidos.
    \end{itemize}
    \item Se calcularon las métricas de precisión global y por persona.
    \item Se construyó la matriz de confusión para visualizar los errores de clasificación.
    \item Se analizaron las distribuciones de distancias para entender el comportamiento del sistema.
\end{enumerate}

% Resultados
\section{Resultados}
\subsection{Resultados del Entrenamiento}
\subsubsection{Análisis de valores singulares}
\subsubsection{Espacio facial generado}
\subsection{Resultados del Reconocimiento}
\subsubsection{Precisión por persona}
\subsubsection{Matriz de confusión}
\subsubsection{Análisis de distancias}
\subsection{Análisis de Compresión}
\subsubsection{Razón de compresión}
\subsubsection{Calidad de reconstrucción}

% Discusión
\section{Discusión}
\subsection{Interpretación de resultados}
\subsection{Comparación con otros métodos}
\subsection{Limitaciones del método}
\subsection{Posibles mejoras}

% Conclusiones
\section{Conclusiones}
\subsection{Logros alcanzados}
\subsection{Trabajo futuro}

% Referencias
\section{Referencias}
\begin{thebibliography}{99}

\end{thebibliography}

% Apéndices
\appendix

\section{Apéndice A: Datos Adicionales}
\subsection{Tablas de Resultados Completos}
\subsection{Gráficos Adicionales}

\end{document}

